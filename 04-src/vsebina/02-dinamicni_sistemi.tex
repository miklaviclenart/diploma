\section{Dinamični sistemi} \label{sec:dis}

V splošnem je dinamični sistem množica stanj skupaj z determinističnim evolucijskim pravilom. Množica stanj je običajno metrični ali topološki prostor \(X\), deterministično evolucijsko pravilo pa preslikava \(F \colon X \to X\). Dinamične sisteme delimo na

\begin{enumerate}
    \item \emph{diskretne} ali \emph{rekurzivne}, kjer je \(x_n \in X\) in \(x_{n + 1} = F (x_n)\);
    \item \emph{zvezne}, ki so sistemi diferencialnih enačb: \(\dot{\mathbf{x}} = F (\mathbf{x})\) za \(\mathbf{x} \in X \subseteq \RR^n\).
\end{enumerate}

\noindent V zgornjih primerih sta indeksni ali \emph{časovni} množici \(\NN\) in \(\RR\). Splošneje je to lahko poljuben monoid (glej \cite{Giunti_2012}).

\begin{zgled}
    Kompleksna števila \(\CC\) skupaj s kompleksno eksponentno preslikavo \(f \colon \CC \to \CC\); \(z \mapsto e^{z}\) so diskretni dinamični sistem.
\end{zgled}

\noindent Označimo \(f^n \coloneq f \circ f^{n - 1}\), kjer je \(f^0 = \Id\).

\begin{definicija}
    Naj bo \(X\) množica in \(f \colon X \to X\) preslikava. \emph{Orbita} začetne točke \(x \in X\) pod preslikavo \(f\) je zaporedje \(\{f^n (x)\}_{n \in \NN}\). Začetni točki pravimo \emph{periodična}, če obstaja tak \(n \in \NN\), da je \(f^n (x) = x\). Če je \(n\) najmanjše tako število, pravimo, da ima periodična točka periodo \(n\). Periodična točka je \emph{fiksna}, če je njena perioda \(\num{1}\).
\end{definicija}

\begin{definicija}[Topološka tranzitivnost]
    Naj bo \((X, \tau)\) topološki prostor in \(f \colon X \to X\) zvezna preslikava. Pravimo, da je \(f\) \emph{topološko tranzitivna}, če za vsaki \(U, V \in \tau\), obstajata \(z \in U\) in \(n \in \NN \cup \set{0}\), da je \(f^n (z) \in V\).
\end{definicija}

\begin{definicija}[Devaneyjev kaos]
    Naj bo \((M, d)\) metrični prostor in \(X \subseteq M\) neskončna. Pravimo, da je zvezna preslikava \(f \colon X \to X\) \emph{kaotična} (po Devaneyju), če sta izpolnjena naslednja pogoja.
    \begin{enumerate}
        \item Množica periodičnih točk je gosta v množici \(X\).
        \item Preslikava \(f\) je topološko tranzitivna.
    \end{enumerate}
\end{definicija}

\begin{zgled}
    Na krožnici \(S^1 \subset \CC\) definiramo rotacijo za kot \(\theta\):
    \[R_\theta \colon S^1 \to S^1; \qquad R_\theta (z) = e^{i \theta} z.\]
    Naj bo \(\theta_1 = \pi\). Potem je vsaka točka periodična s periodo \(\num{2}\), preslikava pa ni topološko  tranzitivna. Naj bo \(\theta_2 = p\), kjer je \(p \in \RR \setminus \QQ\). Potem je preslikava topološko tranzitivna, a niti ena točka ni periodična. Torej je kaos dobro definiran.
\end{zgled}

\begin{zgled}[Podvojitvena preslikava]
    Naj bo \(f \colon [0, 1) \to [0, 1)\) preslikava, podana z
    \[
        f (x) =
        \begin{cases}
            2x & \text{za } 0 \leq x < \frac{1}{2}\\
            2x - 1 & \text{za } \frac{1}{2} \leq x < 1.
        \end{cases}
    \]
    Trdimo, da je \(f\) kaotična.

    Naj bo \(q \in \NN\) liho število in \(A = \set{\frac{1}{q}, \frac{2}{q}, \dots, \frac{q - 1}{q}} \subset [0, 1)\). Potem je \(f \colon A \to A\) bijektivna. Torej je vsak ulomek z lihim imenovalcem periodična točka. Ker so takšni ulomki gosti v \([0, 1)\), ima \(f\) gosto podmnožico periodičnih točk.

    Za dokaz topološke tranzitivnosti si pomagamo z zapisom števila v dvojiški bazi. Opazimo, da preslikava \(f\) ``odreže'' prvo decimalko števila. Definiramo število, ki vsebuje vsa končna zaporedja, urejena po dolžini:
    \[\alpha \coloneq 0, \underbrace{0 1}_{\text{ena}} \underbrace{00011011}_{\text{dva}} \underbrace{000 001 010 100 011 101 110 111}_{\text{tri}} \dots_{(2)}\]
    Naj bo \(x = 0, d_1 d_2 d_3 \dots_{(2)}\) in \(\varepsilon > 0\). Izberemo tak \(n \in \NN\), da je \(2^{- n} < \varepsilon\). Potem obstaja tak \(k \in \NN\), da se število \(f^k (\alpha)\) začne kot \(0, d_1 d_2 \dots d_n d_{n + 1}\), torej da se \(f^k (\alpha)\) in \(x\) začneta razlikovati kvečjemo po \(n + 2\)-ti decimalki. Potem velja
    \[\abs{f^k (\alpha) - x} \leq \sum_{i = n + 2}^{\infty} \frac{2}{2^{i}} < 2^{- n} < \varepsilon.\]
\end{zgled}

\noindent Pojem kaosa je leta \num{1989} definiral ameriški matematik Robert L.~Devaney \cite{Devaney_1986}. Za kompleksno eksponentno preslikavo kot posledica sledi iz izreka  \ref{thm:orbits}. Ta izrek je prvi dokazal poljski matematik Micha\l\ Misiurewicz leta \num{1981} \cite{Misiurewicz_1981}. S tem je potrdil domnevo, ki jo je leta \num{1926} postavil francoski matematik Pierre J.~L.~Fatou \cite{Fatou_1926}.

\subsection{Občutljivost na začetne pogoje}

Devaney je v svoji prvotni definiciji zahteval tudi naslednjo lastnost.

\begin{definicija}[Občutljivost na začetne pogoje]
    Naj bo \((M, d)\) metrični prostor in \(f \colon X \to X\) zvezna preslikava. Pravimo, da je \(f\) občutljiva na začetne pogoje, če obstaja \emph{občutljivostna konstanta} \(\Delta > 0\), da za vsak \(x \in M\) in vsak \(\varepsilon > 0\) obstaja \(y \in M\), da je \(d (x, y) < \varepsilon\) in \(d (f^N (x), f^N (y)) \geq \Delta\) za nek \(N \in \NN\).
\end{definicija}

\noindent To pomeni, da majhna napaka \(\varepsilon\) v začetnih pogojih \(x\) privede do znatnih sprememb pri iteraciji dinamičnega sistema. Vendar pa se izkaže, da te lastnost pri definiciji kaotične preslikave v večini primerov ni potrebno zahtevati. Velja naslednji izrek.

\begin{izrek}
    Kaotična preslikava je občutljiva na začetne pogoje razen na prostoru, ki je sestavljen iz ene periodične orbite.
\end{izrek}

\begin{dokaz}
    Naj bo \(X\) prostor, ki ni zgolj periodična orbita, in \(f \colon X \to X\) kaotična. Ker so periodične točke goste, obstajata vsaj dve različni periodični orbiti. Ker sta disjunktni, obstajata periodični točki \(p, q\), da je \(\Delta \coloneq \min \set{d (f^n (p), f^m (q)) : n, m \in \NN} / 8 > 0\). Pokazali bomo, da je \(\Delta\) občutljivostna konstanta.

    Če je \(x \in X\), potem je orbita ene od točk \(p, q\) vedno oddaljena od \(x\) za vsaj \(4 \Delta\). V nasprotnem primeru, ko bi bile obe v neki točki oddaljene od \(x\) za manj kot \(4 \Delta\), bi bila razdalja med orbitama manjša od \(8 \Delta\). Brez škode za splošnost je to orbita \(q\).

    Naj bo \(\varepsilon \in (0, \Delta)\). Ker so periodične točke goste, obstaja periodična točka \(p \in B (x, \varepsilon)\) s periodo \(n\). Naj bo \(V\) množica točk, ki so po \(n\) iteracijah oddaljene od \(q\) za kvečjemu \(\Delta\):
    \[V \coloneq \bigcap_{i = 0}^n f^{- i} (B (f^i (q), \Delta)).\]
    Po topološki tranzitivnosti obstaja \(k \in \NN\), da je \(f^k (B (x, \varepsilon)) \cap V \neq \emptyset\). Torej obstaja \(y \in B (x, \varepsilon)\), da je \(f^k (y) \in V\). Če definiramo \(j \coloneq \lfloor k / n \rfloor + 1\), potem je \(k / n < j \leq (k / n) + 1\) in
    \[k = n \cdot \frac{k}{n} < n j \leq n \prt{\frac{k}{n} + 1} = k + n.\]
    Torej, če je \(N \coloneq n j\), potem je \(0 < N - k \leq n\). Ker je \(f^N (p) = p\), po trikotniški neenakosti dobimo
    \begin{align}
        d \prt{f^N (p), f^N (y)} &= d \prt{p, f^N (y)} \nonumber\\
        &\geq d \prt{x, f^{N - k} (q)} - d \prt{f^{N - k} (q), f^N (y)} - d (p, x) \label{eqn:ocena}\\
        &\geq 4 \Delta - \Delta - \Delta = 2 \Delta, \nonumber
    \end{align}
    kjer smo upoštevali \(p \in B (x, \varepsilon) \subset B (x, \Delta)\) in
    \[f^N (y) = f^{N - k} \prt{f^k (y)} \in f^{N - k} (V) \subset B \prt{f^{N - k} (q), \Delta},\]
    kar velja po definiciji \(V\). Vemo, da \(p, y \in B (x, \varepsilon)\) in zato po (\ref{eqn:ocena}) velja
    \[d (f^N (p), f^N (x)) \geq \Delta \qquad \text{ali} \qquad d (f^N (y), f^N (x)) \geq \Delta.\]
\end{dokaz}

\noindent Vseeno pa je občutljivost na začetne pogoje pomembna pri obravnavi splošnih dinamičnih sistemov. V javnosti je bolj znana kot \emph{metuljev učinek}. Ime izhaja iz vprašanja, ki ga je meteorolog Ed Lorenz leta \num{1972} postavil na srečanju Ameriškega združenja za napredek znanosti: ``Ali lahko zamah kril metulja v Braziliji povzroči tornado v Teksasu?'' Vprašanje služi kot dobra prispodoba za občutljivost na začetne pogoje, a ga ne smemo vzeti dobesedno, saj ne drži \cite{Pielke_2024}.

\begin{primer}
    Eksponentna funkcija \(f \colon \RR \to \RR\); \(x \mapsto e^x\) je občutljiva na začetne pogoje za običajno metriko. Ker je za vsak \(x \in \RR\) po prvi iteraciji \(e^x > 0\), se lahko omejimo na pozitivna realna števila. Naj bo \(x > 0\) in \(\varepsilon > 0\). Definiramo \(y = x + \varepsilon / 2\). Potem \(d (x, y) = \varepsilon / 2 < \varepsilon\) in
    \[e^{y} - e^{x} = e^{x + \varepsilon / 2} - e^x = e^x \prt{e^{\varepsilon / 2} - 1} > e^{\varepsilon / 2} - 1 > \frac{\varepsilon}{2}.\]
    Zaporedje razdalj med iteracijama \(x\) in \(y\) torej divergira.
\end{primer}

% \noindent Iz definicije je jasno, da je občutljivost na začetne pogoje odvisna od metrike. Zanimivo je vprašanje, ali obstajajo različno metrike, torej da preslikava lastnost ima v eni metriki, a ne v drugi. Iz tega vidika je zanimiva \emph{sferična metrika}. V izogib njeni definiciji bomo navedli le, kdaj je v njej preslikava občutljiva.

% \begin{definicija}
%     Naj bo \(f \colon \CC \to \CC\) zvezna preslikava. Pravimo, da je \(f\) \emph{v sferični metriki občutljiva na začetne pogoje}, če obstajata \(\delta > 0\) in \(R > 0\) z naslednjo lastnostjo. Za vsako neprazno odprto množico \(U \subset \CC\) obstajata \(z, w \in U\) in \(n \geq 0\), tako da velja \(|f^n (z)| \leq R\) in \(|f^n (z) - f^n (w)| \geq \delta\).
% \end{definicija}