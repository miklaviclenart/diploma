\section{Dinamični sistemi} \label{sec:dis}

V splošnem je dinamični sistem množica stanj skupaj z determinističnim evolucijskim pravilom. Množica stanj je običajno metrični ali topološki prostor \(X\), deterministično evolucijsko pravilo pa preslikava \(F \colon X \to X\). Dinamične sisteme delimo na

\begin{enumerate}
    \item \emph{diskretne} ali \emph{rekurzivne}, kjer je \(x_n \in X\) in \(x_{n + 1} = F (x_n)\);
    \item \emph{zvezne}, ki so sistemi diferencialnih enačb: \(\dot{\mathbf{x}} = F (\mathbf{x})\) za \(\mathbf{x} \in X \subseteq \RR^n\).
\end{enumerate}

\noindent V zgornjih primerih sta indeksni ali \emph{časovni} množici \(\NN\) in \(\RR\). Splošneje je to lahko poljuben monoid (glej \cite{Giunti_2012}).

\begin{zgled}
    Kompleksna števila \(\CC\) skupaj s kompleksno eksponentno preslikavo \(f \colon \CC \to \CC\); \(z \mapsto e^{z}\) so diskretni dinamični sistem.
\end{zgled}

\noindent Označimo \(f^n \coloneq f \circ f^{n - 1}\), kjer je \(f^0 = \Id\).

\begin{definicija}
    Naj bo \(X\) množica in \(f \colon X \to X\) preslikava. \emph{Orbita} začetne točke \(x \in X\) pod preslikavo \(f\) je zaporedje \(a_n \coloneq f^n (x)\). Začetni točki pravimo
    \begin{itemize}
        \item \emph{ubežna}, če njena orbita divergira k \(\infty\);\footnote{To pomeni \(\lim_{n \to \infty} |f^n (z)| = \infty\).}
        \item \emph{periodična}, če obstaja tak \(n \in \NN\), da je \(f^n (x) = x\). Če je \(n\) najmanjše tako število, pravimo, da ima periodična točka periodo \(n\);
        \item \emph{fiksna}, če je periodična s periodo \(\num{1}\).
    \end{itemize}
\end{definicija}

\begin{definicija}[Topološka tranzitivnost]
    Naj bo \((X, \tau)\) topološki prostor in \(f \colon X \to X\) preslikava. Pravimo, da je \(f\) \emph{topološko tranzitivna}, če za vsaki \(U, V \in \tau\), obstajata \(z \in U\) in \(n \in \NN \cup \set{0}\), da je \(f^n (z) \in V\).
\end{definicija}

\begin{definicija}[Devaneyjev kaos]
    Naj bo \((M, d)\) metrični prostor in \(X \subseteq M\) neskončna. Pravimo, da je zvezna preslikava \(f \colon X \to X\) \emph{kaotična} (po Devaneyju), če sta izpolnjena naslednja pogoja.
    \begin{enumerate}
        \item Množica periodičnih točk je gosta v množici \(X\).
        \item Preslikava \(f\) je topološko tranzitivna.
    \end{enumerate}
\end{definicija}

\noindent Pojem kaosa je leta \num{1989} definiral ameriški matematik Robert L.~Devaney \cite{Devaney_1986}. Za kompleksno eksponentno preslikavo kot posledica sledi iz izreka  \ref{thm:orbits}. Ta izrek je prvi dokazal poljski matematik Micha\l\ Misiurewicz leta \num{1981} \cite{Misiurewicz_1981}. S tem je potrdil domnevo, ki jo je leta \num{1926} postavil francoski matematik Pierre J.~L.~Fatou \cite{Fatou_1926}.

\begin{zgled}[Podvojitvena preslikava]
    Naj bo \(f \colon [0, 1) \to [0, 1)\) preslikava, podana z
    \[
        f (x) =
        \begin{cases}
            2x & \text{za } 0 \leq x < \frac{1}{2}\\
            2x - 1 & \text{za } \frac{1}{2} \leq x < 1.
        \end{cases}
    \]
    Trdimo, da je \(f\) kaotična.

    Naj bo \(q \in \NN\) liho število in \(A = \set{\frac{1}{q}, \frac{2}{q}, \dots, \frac{q - 1}{q}} \subset [0, 1)\). Potem je \(f \colon A \to A\) bijektivna. Torej je vsak ulomek z lihim imenovalcem periodična točka. Ker so takšni ulomki gosti v \([0, 1)\), ima \(f\) gosto podmnožico periodičnih točk.

    Za dokaz topološke tranzitivnosti si pomagamo z zapisom števila v dvojiški bazi. Opazimo, da preslikava \(f\) ``odreže'' prvo decimalko števila. Definiramo število, ki vsebuje vsa končna zaporedja, urejena po dolžini:
    \[\alpha \coloneq 0, \underbrace{0 1}_{\text{ena}} \underbrace{00011011}_{\text{dva}} \underbrace{000 001 010 100 011 101 110 111}_{\text{tri}} \dots_{(2)}\]
    Naj bo \(x = 0, d_1 d_2 d_3 \dots_{(2)}\) in \(\epsilon > 0\). Izberemo tak \(n \in \NN\), da je \(2^{- n} < \epsilon\). Potem obstaja tak \(k \in \NN\), da se število \(f^k (\alpha)\) začne kot \(0, d_1 d_2 \dots d_n d_{n + 1}\), torej da se \(f^k (\alpha)\) in \(x\) začneta razlikovati kvečjemo po \(n + 2\)-ti decimalki. Potem velja
    \[\abs{f^k (\alpha) - x} \leq \sum_{i = n + 2}^{\infty} \frac{2}{2^{i}} < 2^{- n} < \epsilon.\]
\end{zgled}