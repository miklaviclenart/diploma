\documentclass[a4paper, oneside]{amsart}

\usepackage[utf8]{inputenc}
\usepackage[T1]{fontenc}
\usepackage[slovene]{babel}
\usepackage{lmodern}
\usepackage{microtype}
\usepackage{csquotes}

\usepackage{amsmath, amsthm, amssymb}
\usepackage{mathrsfs}
\usepackage{mathtools}
\usepackage{enumitem}

\usepackage[output-decimal-marker={,}]{siunitx}

\usepackage[
    hyperref=auto,
    isbn=false,
    doi=true,
    url=true,
    date=year,
    giveninits=true,
    maxnames=3,
    %useprefix=true
    backend=biber,
    style=numeric,
    sorting=none]{biblatex}
\addbibresource{literatura.bib}

\usepackage{color}
\definecolor{linkcolor}{RGB}{0,84,147}
\usepackage[
    pdfencoding=auto,
    psdextra,
    colorlinks=true,
    linkcolor=linkcolor,
    citecolor=linkcolor,
    urlcolor=linkcolor,
    filecolor=linkcolor]{hyperref}

\makeatletter
\let\@wraptoccontribs\wraptoccontribs
\makeatother

\title[Kompleksna eksponentna preslikava in kaos]
{Kompleksna eksponentna preslikava in kaos\\Osnutek dela}
\author{Lenart Miklavič}
\address{Fakulteta za matematiko in fiziko, Univerza v Ljubljani}
\contrib[Mentor:]{doc.~dr.~Uroš Kuzman}
\date{}

\newcommand{\NN}{\mathbb{N}}
\newcommand{\ZZ}{\mathbb{Z}}
\newcommand{\QQ}{\mathbb{Q}}
\newcommand{\RR}{\mathbb{R}}
\newcommand{\CC}{\mathbb{C}}
\newcommand{\DD}{\mathbb{D}}

\newcommand{\dd}[1]{\, \mathrm{d} #1}

\DeclareMathOperator{\Id}{Id}

\theoremstyle{plain}
\newtheorem{theorem}{Izrek}[section]
\newtheorem{lemma}[theorem]{Lema}

\theoremstyle{definition}
\newtheorem{definition}[theorem]{Definicija}

\begin{document}

\maketitle

\section{Uvod}
\noindent V diplomski nalogi se ukvarjamo s kaotičnimi lastnostmi preslikave
\(f (z) = e^z\). Obstaja več neekvivalentnih definicij kaotične preslikave.
Najpogosteje uporabljeno je definiral ameriški matematik R.~L.~Devaney
\cite{devaney1989}. Večina mojega dela se tako ukvarja z dokazom, da kompleksna
eksponentna preslikava zadošča pogojem v tej definiciji. Ker je dokaz dolg,
je v tem osnutku naveden zgolj kot skica. Po dokazu sem končni razdelek namenil 
še kratkemu uvodu v splošno transcendentno dinamiko. Večina definicij in izrekov
je vzetih iz osnovnega članka \cite{shen01122015}.

\section{Začetne definicije}
\noindent Pojem kaosa je splošno definiran na poljubnem metričnem prostoru. Ker
pa se ukvarjamo le s preslikavo na kompleksnih številih, lahko predpostavimo
naslednje. Naj bo \(X \subseteq \CC\) množica, \(z \in X\) in  \(f \colon X \to X\)
preslikava. Z \(f^{n}\) označimo \(n\)-to iteracijo preslikave \(f\), definirano
z rekurzivno zvezo \(f^{0} = \Id\) in \(f^{n} = f \circ f^{n - 1}\). Zaporedje
\(a_n \coloneq f^n (z)\) je \emph{orbita} \emph{začetne točke} \(z\) pod
preslikavo \(f\). Začetna točka \(z\) je \emph{periodična}, če obstaja tak
\(n \in \NN\), da je \(f^n (z) = z\). Točka \(z\) je \emph{negibna}, če velja
\(f (z) = z\). Vrednost odvoda \(\lambda = f' (z)\) v negibni točki \(z\)
imenujemo \emph{večkratnost funkcije \(f\) v točki \(z\)}. Glede na večkratnost
funkcije \(\lambda\), je negibna točka:
\begin{itemize}
    \item \emph{superprivlačna}, če je \(\lambda = 0\),
    \item \emph{privlačna}, če je \(|\lambda| < 1\),
    \item \emph{odbojna}, če je \(|\lambda| > 1\),
    \item \emph{racionalno nevtralna}, če je \(|\lambda| = 1\) in
        \(\lambda^n = 1\) za kak \(n \in \NN\),
    \item \emph{iracionalno nevtralna}, če je \(|\lambda| = 1\) in
        \(\lambda^n \neq 1\) za vsak \(n \in \NN\).
\end{itemize}

Preslikava \(f\) je \emph{topološko tranzitivna}, če za vsaki odprti
množici \(U, V \subseteq \CC\), ki sekata \(X\) obstaja tak \(z \in U \cap X\)
in \(n \geq 0\), da je \(f^n (z) \in V\). Sedaj lahko definiramo pojem kaosa.

\begin{definition}[Devaneyev kaos]
    Naj bo \(X \subseteq \CC\) neskončna množica in \(f \colon X \to X\) zvezna.
    Pravimo, da je \(f\) \emph{kaotična} če velja naslednje.
    \begin{enumerate}
        \item Množica periodičnih točk \(f\) je gosta v \(X\).
        \item Preslikava \(f\) je topološko tranzitivna.
    \end{enumerate}
\end{definition}

V svoji prvotni definiciji je Devaney zahteval tudi \emph{občutljivost na začetne
pogoje}\footnote{Zvezna preslikava \(f \colon X \to X\) na metričnem prostoru
\((X, d)\) je občutljiva na začetne pogoje, če obstaja tak \(\delta > 0\), da za
vsako odprto množico \(U \subset X\) obstajata \(x, y \in U\), da velja
\(d (f^n(x), f^n (y)) > \delta\) za nek \(n \in \NN\).}, vendar se izkaže, da
za zvezne preslikave na neskončnih metričnih prostorih, ki izpolnjujejo prva dva
pogoja, to vedno velja \cite{implikacija1992}.

Če se omejili zgolj na realno eksponentno funkcijo (\(X = \RR\)), velja, da so vse
točke ubežne. Enako ne velja za \(X = \CC\). Glavni rezultat mojega dela je 
torej naslednji izrek.

\begin{theorem}
    Kompleksna eksponentna preslikava je kaotična.
\end{theorem}

Pri dokazovanju bomo uporabili tudi kompleksno logaritemsko preslikavo. Ker
kompleksna eksponentna preslikava ni injektivna, je za kompleksni števili
\(z\) in \(w\) rešitev enačbe \(e^w = z\) neskončno mnogo:
\[w = \ln |z| + i \operatorname{arg} z + 2 k \pi i,\]
pri čemer je \(k \in \ZZ\) in za \(\theta \in \RR\):
\(\operatorname{arg} z \in [\theta - \pi, \theta + \pi)\). Vsaki taki funkciji
rečemo \emph{veja logaritma}.

Dokaz izreka je razdeljen na tri dele. V prvem delu dokažemo, da je množica
\emph{ubežnih točk} (točk, katerih orbita divergira), gosta v kompleksni ravnini.
Kot zanimivost bo iz tega takoj sledila občutljivost na začetne pogoje. V drugem
delu bomo gostost ubežnih točk uporabili pri dokazu topološke tranzitivnosti.
V tretjem delu pa dokažemo še gostost periodičnih točk.

\section{Hiperbolična geometrija}
\noindent Pri dokazovanju bomo uporabili nekaj osnovnih in lahko prevevrljivih
dejstev iz \emph{hiperbolične geometrije}. Z \(\DD\) označimo enotski disk,
torej \(\DD \coloneq \{ z \in \CC : |z| < 1 \}\). Iz Analize 2b poznamo naslednjo
lemo.

\begin{lemma}[Schwarz]
    Naj bo \(f \colon \DD \to \DD\) holomorfna preslikava, za katero velja
    \(f (0) = 0\). Potem velja ena od naslednjih možnosti.
        \begin{enumerate}
            \item \(|f (z)| < |z|\) za vsak neničelni \(z \in \DD\) in \(|f' (0)| < 1\).
            \item Obstaja \(\theta \in \RR\) da za vsak \(z\) velja
                \(f (z) = e^{i \theta} z\) in \(|f' (0)| = |e^{i \theta}| = 1\).
        \end{enumerate}
\end{lemma}

Lemo bi radi uporabili, vendar za eksponentno funkcijo predpostavke niso
izpolnjene. Zato našo preslikavo transformiramo M\"obiusovo transformacijo in
njenim inverzom, uporabimo verižno pravilo za odvajanje kompozituma, da se lema
prevede na
\[|f' (z)| \cdot \frac{1 - |z|^2}{1 - |f (z)|^2} \leq 1 \qquad \text{za vsak } z \in \DD.\]
Rezultat nam pove, da je odvod preslikave \(f\) največ \(1\), če ga izračunamo
``glede na drugačno metriko''.

% Bolj natančno rečemo izrazu
% \[\frac{2 |\mathrm{d} z|}{1 - |z|^2}\]
% hiperbolična metrika

\begin{definition}
    Naj bo \(\gamma \colon [a, b] \to \DD\) odsekoma \(C^1\) krivulja. Njena
    \emph{hiperbolična dolžina} je
    \[l_{\DD} (\gamma) \coloneq \int_{a}^{b} \frac{2 |\gamma' (t)|}{1 - |\gamma (t)|^2} \dd{t}.\]
    \emph{Hiperbolična razdalja} med točkama \(z, w \in \DD\) je
    \[d_{\DD} (z, w) \coloneq \inf_{\gamma} l_{\DD} (\gamma),\]
    kjer \(\gamma\) teče po vseh odsekoma \(C^1\) krivuljah v \(\DD\), ki
    povezujejo točki \(z\) in \(w\).
\end{definition}

Izkaže se, da je metrika dobro definirana. Za nas bo pomembena \emph{gostota}
hiperbolične metrike
\[\rho_{\DD} (z) \coloneq \frac{2}{1 - |z|^2},\]
ki je faktor, s katerim pomnožimo infinitezimalno spremembo v evklidski metriki,
da dobimo hiperbolično. Z indeksom \(\DD\) poudarimo, da izraz velja le na disku.
Z naslednjim izrekom metriko definiramo še na poljubnem območju.

% \begin{lemma}[Schwarzeva lema]
%     Naj bo \(f \colon \DD \to \DD\) holomorfna preslikava in \(f (0) = 0\).
%     Potem velja ena od naslednjih možnosti.
%     \begin{enumerate}
%         \item \(|f (z)| < |z|\) za vsak neničelni \(z \in \DD\) in \(|f' (0)| < 1\).
%         \item Obstaja \(\theta \in \RR\) da za vsak \(z\) velja
%             \(f (z) = e^{i \theta} z\) in \(|f' (0)| = |e^{i \theta}| = 1\).
%     \end{enumerate}
% \end{lemma}

% Lemo posplošimo na poljubno holomorfna preslikavo z \emph{M\"obiusovo
% transformacijo}
% \[M \colon \DD \to \DD; \qquad z \mapsto e^{i \theta} \frac{z - a}{1 - \bar{a} z},\]
% kjer je \(\theta \in \RR\) poljuben. Na poljubni holomorfni funkciji
% \(f \colon \DD \to \DD\) sedaj uporabimo transformacijo in njen inverz ter stvar
% odvajamo po verižnem pravilu in dobimo neenakost
% \[|f' (z)| \cdot \frac{1 - |z|^2}{1 - |f (z)|^2} \leq 1 \qquad \text{za vsak } z \in \DD.\]
% Ta neenakost nam pove, da je vrednost odvoda \(f\) kvečjemo \(1\)

% Najbolj pomembna stvar od tuki je, da holomorfna preslikava \(f \colon U \to V\)
% ne širi hiperbolične razdalje, kar pomeni:
% \[|f' (z)| \cdot \frac{\rho_V (f ( z))}{\rho_U (z)} \leq 1.\]

% Naj bo \(U \subseteq \CC\) odprta množica in \(f \colon U \to \CC\).
% Pravimo, da je \(f\) \emph{konformna}, če je holomorfna in je njen odvod
% na \(U\) različen od nič.

\begin{theorem}[Pick] \label{pick}
    Za vsako območje \(U \subsetneq \CC\) obstaja enolična konformna\footnote{Naj
    bo \(U \subseteq \CC\) odprta množica in \(f \colon U \to \CC\). Pravimo, da
    je \(f\) \emph{konformna}, če je holomorfna in je njen odvod na \(U\) različen
    od nič.} metrika \(\rho_U (z) |\mathrm{d} z|\) na \(U\), ki ji rečemo
    \emph{hiperbolična metrika} in izpolnjuje naslednje:
    \begin{enumerate}
        \item \(\rho_\DD (z) = \frac{2}{1 - |z|^2}\) za vsak \(z \in \DD\);
        \item če je \(f \colon U \to V\) holomorfna, potem \(f\) ne veča
            hiperbolične razdalje; to je
            \[\left\| \mathrm{D} f (z) \right\|_U^V \coloneq \left| f' (z) \right| \cdot \frac{\rho_V (f (z))}{\rho_U (z)} \leq 1;\]
        \item za vsak \(z \in U\) in vsako \(f\) kot zgoraj velja
            \(\|\mathrm{D} f (z)\|_U^V = 1\) natanko tedaj, ko je \(f\) konformni
            izomorfizem med \(U\) in \(V\);
        \item če je \(U \subsetneq V\), potem je \(\rho_U (z) > \rho_V (z)\)
            za vsak \(z \in U\).
    \end{enumerate}
\end{theorem}


\section{Gostost ubežnih točk}
\noindent Naj bo \(D \subset \CC\) majhen disk. Ker je
\(\RR \subset I (f)\) in je vsaka praslika ubežne točke tudi ubežna, trditev
velja, če \(D\) vsebuje točko, katere orbita vsebuje realno število.
Drugače pa so \(D, f(D), \dots, f^n (D)\) vsebovane v
\[U \coloneq \CC \setminus [0, \infty).\]
Opazimo, da za \(U\) velja \(f^{-1} (U) \subseteq U\). Od tod sledi, da je vsaka
veja \(L\) logaritma na \(U\) holomorfna preslikava \(L \colon U \to U\) in zato
lokalno krči hiperbolično metriko na \(U\).

Dokazujemo s protislovjem. Predpostavimo \(D \cap I (f) = \emptyset\). Potem ima
zaporedje domen \(f^n (D)\) vsaj eno končno stekališče. Izkaže se, da to zadosten
pogoj, da \(f\) širi hiperbolično metriko za določen faktor neskončno krat.
Po drugi strani pa je po Pickovem izreku hiperbolični odvod po \(U\) omejen, ko
gre \(n \to \infty\). Prišli smo do protislovja.

\section{Topološka tranzitivnost}
\noindent Uporabimo gostost ubežnih točk. Vemo, da \(f\) po orbiti vsake ubežne
točke \(z_0\) strogo narašča. Za dovolj velike \(n\) iteracija \(f^n\) poljubno
majhen disk s središčem v \(z_0\) preslika v množico, ki vsebuje disk z radijem
\(2 \pi\) in središčem v \(f^n (z_0)\). Če takšen disk še dvakrat preslikamo
z \(f\), se ta razširi čez velik del kompleksne ravnine. Kot posledico tega
izpeljemo nekoliko močnejšo trditev, da za vsak kompakt
\(K \subset \CC \setminus \{0\}\) in vsako odprto množico \(U \subset \CC\)
obstaja \(N \in \NN\), tako da je \(K \subset f^n (U)\) za vsak \(n \geq N\).

\section{Gostost periodičnih točk}
\noindent Izkaže se, da je vsaka periodična točka odbojna. Tako dokazujemo, da
je množica odbojnih periodičnih točk gosta. Naj bo \(U \subset \CC\) odprta
in neprazna. Ker je množica ubežnih točk gosta, obstaja ubežna točka
\(z_0 \in U \setminus \{ 0 \}\). Naj bo \(\Delta\) disk s središčem
v \(z_0\). Poščemo vejo
logaritma iteracije \(f\), ki preslika disk \(\Delta\) nazaj samega vase in ga
pri tem skrči. Obstoj periodične točke nato sledi po Banachovem skrčitvenem načelu.

\section{Uvod v transcendentno dinamiko}
\noindent Dokazali smo, da je eksponentna preslikava kaotična na celotni
kompleksni ravnini  in zato tam tudi občutljiva na začetne pogoje, kar sledi iz
rezultata omenjenega
v drugem razdelku. Za splošno holomorfno preslikavo \(f \colon \CC \to \CC\) pa
to ne drži. Množico točk \(J (f)\), kjer je \(f\) občutljiva na začetne pogoje
imenujemo \emph{Juliajeva množice}. Njen komplement \(F (f) = \CC \setminus J (f)\),
kjer je \(f\) stabilna, pa imenujemo \emph{Fatoujeva množica}.
Bolj natančno definiramo Fatoujevo množico z \emph{normalnimi družinami}.

\begin{definition}[Normalne družine]
    Naj bo \(\mathcal{C} (X, Y)\) množica zveznih preslikav med topološkima 
    prostoroma \(X, Y\), ki jo opremimo s kompaktno-odprto topologijo.
    Relativno kompaktni podmnožici tega prostora rečemo \emph{normalna družina}.
\end{definition}

V kompleksni ravnini je družina \(\mathcal{F}\) meromorfnih funkcij na območju
\(\Omega \subset \CC\) normalna natanko tedaj, ko vsako zaporedje
\(\{f_n\} \subset \mathcal{F}\) konvergentno podzaporedje v topologiji enakomerne
konvergence na kompaktih. Velja tudi naslednji izrek.

\begin{theorem}[Montelov izrek]
    Družina \(\mathcal{F}\) holomorfnih funkcij na \(\Omega \subset \CC\), ki je
    na vsakem kompaktu omejena z neko fiksno konstanto, je normalna.
\end{theorem}

Sedaj lahko navedemo natančno definicijo.

\begin{definition}[Fatoujeva in Juliajeva množica]
    Naj bo \(f \colon \CC \to \CC\) cela funkcija in \(\{ f^n \}\) družina njenih
    iteracij. \emph{Fatoujeva množica} je
    \[F (f) = \{ z \in \CC : \{ f^n \} \text{ je normalna družina na neki okolici } z \},\]
    njenemu komplementu \(J (f) = \CC \setminus F (f)\) pa je \emph{Juliajeva}
    množica.
\end{definition}

\begin{theorem}[Kaos na Juliajevi množici]
    Juliajeva množica \(J (f)\) je vedno neštevno neskončna in \(f^{-1} (J(f)) = J(f)\).
    Funkcija \(f \colon J (f) \to J (f)\) je kaotična.
\end{theorem}

Komponente za povezanost Fatoujeve množice imenujemo \emph{Komponente Fatoujeve
množice}. Naj bo \(U\) komponenta \(F (f)\). Če obstaja \(n\), tako da
\(f^n (U) = U\), potem je \(U\) \emph{periodična komponenta s periodo \(p\)}. 

\begin{theorem}[Klasifikacija Fatoujevih komponent \cite{bergweiler1993}]
    Naj bo \(U\) periodična komponenta s periodo \(p\).
    Potem velja eno od naslednjih možnosti.
    \begin{enumerate}
        \item \(U\) vsebuje privlačno točko \(z_0\) periode \(p\). Potem
            \(f^{np} (z) \to z_0\) za \(z \in U\) ko gre \(n \to \infty\).
            % \(U\) v tem primeru imenujemo \emph{immediate attracting basin} točke
            % \(z_0\).
        \item Rob \(U\) vsebuje periodično točko \(z_0\) periode \(p\) in
            \(f^{np} (z) \to z_0\) za \(z \in U\) ko gre \(n \to \infty\).
            Potem \((f^p)' (z_0) = 1\) če \(z_0 \in \CC\). (Za \(z_0 = \infty\)
            velja \((g^p)' (0) = 1\) kjer je \(g (z) 1 / f (1 / z)\).)
            V tem primeru \(U\) imenujemo \emph{Leaujeva domena}.
        \item Obstaja analitičen homeomorfizem \(\phi \colon U \to \DD\), tako da
            velja \(\phi (f^p (\phi^{-1} (z))) = e^{2 \pi i \alpha} z\) za nek
            \(\alpha \in \RR \setminus \QQ\). V tem primru \(U\) imenujemo
            \emph{Siegelov disk}.
        \item Obstaja analitičen homeomorfizem \(\phi \colon U \to A\) kjer je
            \(A = \{ z : 1 < |z| < r \}\) (\(r > 0\)) krožni kolobar, tako da
            velja \(\phi (f^p (\phi^{-1} (z))) e^{2 \pi i \alpha} z\) za nek
            \(\alpha \in \RR \setminus \QQ\). V tem primeru \(U\) imenujemo
            \emph{Hermanov kolobar}.
        \item Obstaja \(z_0 \in \partial U\) tako da \(f^{np} (z) \to z_0\)
            za \(z \in U\) ko gre \(n \to \infty\), vendar \(f^p (z_0)\) ni
            definirana. V tem primeru \(U\) imenujemo \emph{Bakerjeva domena}.
    \end{enumerate}
\end{theorem}

\printbibliography
\end{document}